\documentclass{amsart}
\usepackage{math}
\begin{document}

Pour commencer quelque part: j'ai regard\'e les d\'efinitions de ``smale space'', et essay\'e de les rendre uniformes. Mon premier jet est ci-dessous.

Je refl\'echis d\'ej\`a \`a la syntaxe de Lean.

(1) il me semble que c'est une bonne id\'ee de d'abord parler d'un espace avec crochet, et ensuite d'un hom\'eomorphisme. \c{C}a double le nombre de d\'efinitions, mais \c{c}a les rend plus courtes.

(2) Je ne connais pas tr\`es bien la notation pour les structures uniformes. J'aimerais bien dire ``pour tous $x,y$ assez proches'', ce qui veut dire ``il existe un entourage $\mathcal U$ tel que pour tout $(x,y)\in\mathcal U$''; est-ce qu'on peut dire \c{c}a de mani\`ere plus compacte? Pour moi, les entourages sont toujours ouverts, j'esp\`ere que c'est le cas pour tout le monde.

(3) qui est une cons\'equence de (2), il me semble bien de dire que le crochet est d\'efini partout, mais qu'il est continu seulement pr\`es de la diagonale, qu'il est pr\'eserv\'e par $f$ seulement pr\`es de la diagonale, etc. \c{C}a va peut-\^etre \'eviter un peu de yoga de coercion. L'inconv\'enient est qu'il faut le d\'efinir de mani\`ere d\'ebile (comme $x/0=0$) loin de la diagonale.

(4) qui est une cons\'equence de (2), deux alg\`ebres de projection sont isomorphes si leurs crochets correspondent sur un entourage.

\begin{defn}
  Une \emph{ProjectionAlgebra} est
  \begin{itemize}
  \item un espace uniforme $X$;
  \item un ``crochet de Ruelle'' $[,]\colon X^2\to X$
  \end{itemize}
  tels que:
  \begin{enumerate}
  \item il existe un entourage $\mathcal U$ sur lequel $[,]$ est uniform\'ement continu;
  \item $[x,x]=x$;
  \item il existe un entourage $\mathcal U$ tel que, pour tous $x,y,z$ avec $(x,y),(x,z),(y,z)\in\mathcal U$ on a
    \[[[x,y],z]=[x,[y,z]]=[x,z].\]
  \end{enumerate}
\end{defn}

\begin{lem}
  Pour tout $n$ il existe un entourage $\mathcal U_n$ tel que, pour tous $x_1,\dots,x_n$ avec $\forall i,j:(x_i,x_j)\in\mathcal U$, on a
  \[[x_1,x_2,\dots,x_n]=[x_1,x_n]\text{ quel que soit l'ordre des crochets au milieu}.\]
\end{lem}

On d\'efinit, pour un entourage $\mathcal V$, les sous-ensembles de $X$
\[\mathcal V_x^-=\{u:u=[u,x],(u,x)\in\mathcal V\},\quad\mathcal V_x^+=\{u:u=[x,u],(u,x)\in\mathcal V\}.\]
\begin{lem}
  Il existe un entourage $\mathcal V$ tel que, pour tout $x$, la restriction de $[,]$ \`a $\mathcal V_x^-\times\mathcal V_x^+$ est un hom\'eomorphisme sur son image.
\end{lem}
\begin{proof}
  On prend $\mathcal V$ assez fin pour que $(x,u),(x,v)\in\mathcal V\Rightarrow (x,[u,v])\in\mathcal U_4$.

  \'Etant donn\'es $(x,y)\in\mathcal U_4$, on pose $u=[y,x],v=[x,y]$, et alors
  \[u=[u,x],v=[x,v],[u,v]=y,[v,u]=x.\]
  L'inverse de $[,]\colon\mathcal V_x^-\times\mathcal V_x^+\to X$, est donc donn\'e par $y\mapsto([y,x],[x,y])$.
\end{proof}
\begin{cor}
  Si $(x,y)\in\mathcal U_4$ alors $\{[x,y]\}=\mathcal V_x^+\cap\mathcal V_y^-$ avec $\mathcal V=\mathcal U_3$.
\end{cor}

\begin{defn}
  Un \emph{SmaleSpace} est
  \begin{itemize}
  \item une ProjectionAlgebra $X$;
  \item un hom\'eomorphisme uniform\'ement continu $f\colon X\to X$
  \end{itemize}
  tels que:
  \begin{enumerate}
  \item il existe un entourage $\mathcal U$ avec $\forall(x,y)\in\mathcal U:f([x,y])=[f(x),f(y)]$;
  \item il existe un entourage $\mathcal U$ tel que, pour tout entourage $\mathcal V\subset\mathcal U$, il existe un entourage $\mathcal W\subset\mathcal V$ ``beaucoup plus petit'' (``$\lambda$ fois plus petit''), avec
    \[\forall x:f(\mathcal V_x^+)\subseteq \mathcal W_{f(x)}^+,\quad f^{-1}(\mathcal V_x^-)\subseteq\mathcal W_{f(x)}^-.\]
  \end{enumerate}
\end{defn}

\begin{lem}[Putnam, Theorem 2.2.2]
  Pour tout entourage $\mathcal U$ assez petit, et pour tout entier $N$, il existe un entourage $\mathcal U_N$ avec $\forall(x,y)\in\mathcal U_N,\forall i\in\{0,\dots,N\}:f^i([x,y])=[f^i(x),f^i(y)]\in\mathcal U$.
\end{lem}

\subsection{Exemple 1}
on prend un ensemble $A$, qu'on voit comme espace topologique discret, et on consid\`ere $X_0=A^{\mathbb Z}$ avec la topologie produit, et les entourages engendr\'es par les $\mathcal U_{n,m}=\{(x,y)\in X_0^2:x_t=y_t\forall t\in[m,n]\}$ avec $m,n\in\mathbb Z$. L'application $f$ est le d\'ecalage. Finalement on se donne un graphe dirig\'e $G\subseteq A^2$, et on regarde $X=\{x\in X_0:(x_n,x_{n+1}\in G\forall n\}$.

Alors $[x,y]=(\dots,y_{-1},y_0,x_1,x_2,\dots)$ pour $(x,y)\in\mathcal U_{0,0}$, et $f^{\pm1}$ contracte $\mathcal U_{0,0}$ en $\mathcal U_{-1,1}$ respectivement sur $(\mathcal U_{0,0})_x^{\pm1}$.

\subsection{Tra\c{c}age}

\begin{thm}[Ombach, Theorem]
  Soit $X$ uniforme, $f\colon X\to X$ un hom\'eomorphisme uniforme. TFAE:
  \begin{enumerate}
  \item $f$ a des coordonn\'ees canoniques et est expansif;
  \item $f$ a des coordonn\'ees hyperboliques wrt une m\'etrique [probablement seulement si $X$ est m\'etrisable];
  \item $f$ trace les pseudo-orbites et est $\lambda$-expansif wrt une m\'etrique [probablement seulement si $X$ est m\'etrisable compact];
  \item $f$ trace les pseudo-orbites et est expansif;
  \item $f$ a des coordonn\'ees r\'eguli\`eres;
  \item $f$ est un espace de Smale.
  \end{enumerate}
\end{thm}

\begin{defn}
  \[W^u=\{(x,y)\in X^2:f^n(x,y)\to\Delta\text{ quand }n\to-\infty\}.\]
  \[W^s=\{(x,y)\in X^2:f^n(x,y)\to\Delta\text{ quand }n\to+\infty\}.\]
  Pour un entourage $\mathcal U$,
  \[W_{\mathcal U}^u=\bigcap_{n\le0}f^n(\mathcal U),\quad
    W_{\mathcal U}^s=\bigcap_{n\ge0}f^n(\mathcal U).\]

  Si $U\subseteq X^2$ est une relation, on note $U(x)=\{y: (y,x)\in U\}$, etc.

  $f$ a des \emph{coordonn\'ees canoniques} si
  \[\forall\mathcal U\text{ assez petit}:\exists\mathcal V:(x,y)\in\mathcal V\Rightarrow W_{\mathcal V}^s(x)\cap W_{\mathcal V}^u(y)\ne\emptyset.\]

  $f$ a des \emph{coordonn\'ees r\'eguli\`eres} si
  \[\forall\mathcal U\text{ assez petit}:\exists\mathcal V:\exists[,]\colon\mathcal V\to X\text{ continue}:(x,y)\in\mathcal V\Rightarrow W_{\mathcal V}^s(x)\cap W_{\mathcal V}^u(y)=\{[x,y]\},\]
  et $W_{\mathcal U}^s\subset W^s,W_{\mathcal U}^u\subset W^u$.

  $f$ a des \emph{coordonn\'ees hyperboliques} s'il existe un entourage $\mathcal U$, $\lambda\in(0,1)$, $C\ge0$ tq
  \[W_{\mathcal U}^s\subset\{(x,y):d(f^n x,f^n y)\le C\lambda^n d(x,y)\forall n\ge0\},\]
  \[W_{\mathcal U}^u\subset\{(x,y):d(f^n x,f^n y)\le C\lambda^n d(x,y)\forall n\le0\}.\]

  $f$ est \emph{expansive} s'il existe un entourage $\mathcal U$ tel que $\bigcap_{n\in\mathbb Z}f^n(\mathcal U)=\Delta$.

  $f$ est \emph{$\lambda$-expansive} avec $\lambda\in(0,1)$ s'il existe $C>0,e>0$ tels que
  \[\forall n\in\mathbb N: (d(f^k x,f^k y)\le e\forall |k|\le n)\Rightarrow d(x,y)\le C\lambda^n d(x,y).\]

  $f$ \emph{trace les pseudo-orbites} si pour tout entourage $\mathcal U$ il existe un entourage $\mathcal V$ tel que toute $\mathcal V$-pseudo-orbite est $\mathcal U$-trac\'ee par une orbite; i.e. si $(x_n)_{n\in\mathbb Z}$ satisfait $\forall n:(x_n,x_{n+1})\in\mathcal V$, alors il existe $x$ avec $\forall n:(f^n(x),x_n)\in\mathcal U$.
\end{defn}
\begin{proof}

  [Bowen, Proposition 3.6 et Corollary 3.7]

  [Ombach]
\end{proof}

\subsection{D\'ecomposition spectrale}

[Bowen, Theorem 3.5]

\subsection{Partitions de Markov}

[Bowen, Theorem 3.12]

\subsection{R\'ef\'erences}

I. Putnam, Lecture Notes on Smale Spaces, 2015.

R. Bowen. Equilibrium States and the Ergodic Theory of Anosov diffeomorphisms. Lecture Notes in Math. No. 470. Springer, Berlin, 1975.

J. Ombach, Equivalent conditions for hyperbolic coordinates, Topology Appl. 23 (1986) 87-90.

\end{document}
